\chapter{ExRORU的实现与实验}\label{cha:experiment}

\section{ExRORU算法的实现}\label{sec:implementation}

\section{实验设计与分析}\label{sec:experiment}
本文实现的ExRORU算法基于jbpt\footnote{\url{https://code.google.com/p/jbpt}}中的CPU抽取WF-net的ExRORU关系并以矩阵形式展现,相关代码已开源共享在Github上\footnote{\url{https://github.com/shudiwsh2009/ExRORU}}。本节介绍ExRORU的实验设计与分析,主要包括有效性实验(将ExRORU与其他算法比较区分过程模型的能力)、性能实验(使用实际模型数据集衡量ExRORU算法的效率)和扩展性实验(衡量ExRORU处理含多并发结构过程模型的能力)等。

\subsection{有效性实验}\label{subsec:effectiveness}
本小节主要展示ExRORU算法强大的刻画过程模型行为语义的能力,并通过在多个模型实例中与第\ref{cha:related_work}中介绍的算法对比,以说明其区分含有不同行为语义的过程模型的能力。

{\heiti 非自由选择结构。}过程模型的任务之间存在间接依赖关系\cite{van2004workflow,van2003workflow,de2003workflow,van2004process},在WF-net上造成了非自由选择结构的存在(即选择关系与同步关系混合的情况)。图\ref{fig:nfc_example_1}中的模型是一个典型的含有非自由选择结构的模型,其中变迁$D$和变迁$E$之间存在非自由选择情形,即他们的执行并不由自身所决定,而是由之前被执行的变迁$A$和变迁$B$决定。具体分析,当在该WF-net的源库所$P_{0}$中放置一个托肯时,变迁$A$和变迁$B$同时被使能,若此时选择执行变迁$A$,则会在库所$_{1}$和库所$P_{2}$中各产生一个托肯;此时只有变迁$C$被使能,其发生之后会消耗库所$P_{1}$的托肯,同时在库所$P_{4}$中产生一个托肯;库所$P_{2}$和库所$P_{4}$各有一个托肯时,显然只有变迁$D$被使能。另一方面,若一开始选择执行变迁$B$,同理可知在变迁$C$发生之后只有变迁$E$被使能。因此,该模型的变迁执行序列只有$\langle A,C,D\rangle$和$\langle B,C,E\rangle$两条。

与之相对的是图\ref{fig:nfc_example_2}中不含有非自由选择结构的模型。与图\ref{fig:nfc_example_1}中的模型相比,该模型少了两个库所和四条边,其中变迁$D$和变迁$E$不存在非自由选择情形,即他们的执行由自身竞争库所$P_{2}$中的托肯所决定。因此,该模型的变迁执行序列有$\langle A,C,D\rangle$、$\langle A,C,E\rangle$、$\langle B,C,D\rangle$和$\langle B,C,E\rangle$四条。

以TAR算法为例,两个模型的TAR集合都是$\{\langle A,C\rangle,\langle C,D\rangle,\langle B,C\rangle,\langle C,E\rangle\}$,显然TAR算法未考虑非自由选择结构蕴含的任务间接依赖关系,所以无法区分这一组模型的行为。本文改进的基础RORU算法是通过任务间关系传递性来抽取因果关系的,然而在图\ref{fig:nfc_example_1}的模型中,变迁$A$与$C$之间的因果关系和变迁$C$与$E$之间的因果关系并不满足传递性,故RORU算法在处理该模型时会出错。

实际上,图\ref{fig:nfc_example_1}的模型只有两个过程流,分别表示为$[A\{C\}D]$和$[B\{C\}E]$(虽然在变迁$A$和变迁$B$后都有并行结构,但是均只有一个分支含有变迁$C$,另一个分支是无变迁分支)。因此,在该模型中,变迁$A$和变迁$D$满足“间接总是因果关系”,即$A\overset{\text{\tiny{IA}}}{\rightarrow}D$;变迁$B$和变迁$E$也满足“间接总是因果关系”,即$B\overset{\text{\tiny{IA}}}{\rightarrow}E$。另一方面,图\ref{fig:nfc_example_2}的模型有4个过程流,分别表示为$[ACD]$、$[ACE]$、$[BCD]$和$[BCE]$,因此在该模型中变迁$A$和变迁$D$满足“间接有时因果关系”,即$A\overset{\text{\tiny{IS}}}{\rightarrow}D$;变迁$B$和变迁$E$也满足“间接有时因果关系”,即$B\overset{\text{\tiny{IS}}}{\rightarrow}E$。两个模型的ExRORU关系矩阵如表\ref{tab:nfc_example}所示,两个模型的行为差异主要体现在变迁$A$与变迁$D$、变迁$E$之间的因果关系和逆因果关系以及变迁$B$与变迁$D$、变迁$E$之间的因果关系和逆因果关系上。因此,ExRORU算法可以检测该组模型之间的差异。

\begin{table}[htbp]
  \centering
  \setlength\tabcolsep{4pt}
  \caption{图\ref{fig:nfc_example}中两个模型的ExRORU矩阵}
  \label{tab:nfc_example}
  \begin{subtable}{1\textwidth}
    \centering
    \caption{图\ref{fig:nfc_example_1}中模型的ExRORU矩阵}
    \label{tab:nfc_example_1}
    \begin{minipage}[b]{0.3\textwidth}
      \centering
      \begin{tabular}{|c|c|c|c|c|c|} \hline
        $\rightarrow$ & $A$ & $B$ & $C$ & $D$ & $E$\\ \hline
        $A$ & $\overset{\text{\tiny{N}}}{\rightarrow}$ & $\overset{\text{\tiny{N}}}{\rightarrow}$ & $\overset{\text{\tiny{DA}}}{\rightarrow}$ & $\overset{\text{\tiny{DA}}}{\rightarrow}$ & $\overset{\text{\tiny{N}}}{\rightarrow}$\\ \hline
        $B$ & $\overset{\text{\tiny{N}}}{\rightarrow}$ & $\overset{\text{\tiny{N}}}{\rightarrow}$ & $\overset{\text{\tiny{DA}}}{\rightarrow}$ & $\overset{\text{\tiny{N}}}{\rightarrow}$ & $\overset{\text{\tiny{DA}}}{\rightarrow}$\\ \hline
        $C$ & $\overset{\text{\tiny{N}}}{\rightarrow}$ & $\overset{\text{\tiny{N}}}{\rightarrow}$ & $\overset{\text{\tiny{N}}}{\rightarrow}$ & $\overset{\text{\tiny{DS}}}{\rightarrow}$ & $\overset{\text{\tiny{DS}}}{\rightarrow}$\\ \hline
        $D$ & $\overset{\text{\tiny{N}}}{\rightarrow}$ & $\overset{\text{\tiny{N}}}{\rightarrow}$ & $\overset{\text{\tiny{N}}}{\rightarrow}$ & $\overset{\text{\tiny{N}}}{\rightarrow}$ & $\overset{\text{\tiny{N}}}{\rightarrow}$\\ \hline
        $E$ & $\overset{\text{\tiny{N}}}{\rightarrow}$ & $\overset{\text{\tiny{N}}}{\rightarrow}$ & $\overset{\text{\tiny{N}}}{\rightarrow}$ & $\overset{\text{\tiny{N}}}{\rightarrow}$ & $\overset{\text{\tiny{N}}}{\rightarrow}$\\ \hline
      \end{tabular}
    \end{minipage}
    \begin{minipage}[b]{0.3\textwidth}
      \centering
      \begin{tabular}{|c|c|c|c|c|c|} \hline
        $\leftarrow$ & $A$ & $B$ & $C$ & $D$ & $E$\\ \hline
        $A$ & $\overset{\text{\tiny{N}}}{\leftarrow}$ & $\overset{\text{\tiny{N}}}{\leftarrow}$ & $\overset{\text{\tiny{N}}}{\leftarrow}$ & $\overset{\text{\tiny{N}}}{\leftarrow}$ & $\overset{\text{\tiny{N}}}{\leftarrow}$\\ \hline
        $B$ & $\overset{\text{\tiny{N}}}{\leftarrow}$ & $\overset{\text{\tiny{N}}}{\leftarrow}$ & $\overset{\text{\tiny{N}}}{\leftarrow}$ & $\overset{\text{\tiny{N}}}{\leftarrow}$ & $\overset{\text{\tiny{N}}}{\leftarrow}$\\ \hline
        $C$ & $\overset{\text{\tiny{DS}}}{\leftarrow}$ & $\overset{\text{\tiny{DS}}}{\leftarrow}$ & $\overset{\text{\tiny{N}}}{\leftarrow}$ & $\overset{\text{\tiny{N}}}{\leftarrow}$ & $\overset{\text{\tiny{N}}}{\leftarrow}$\\ \hline
        $D$ & $\overset{\text{\tiny{DA}}}{\leftarrow}$ & $\overset{\text{\tiny{N}}}{\leftarrow}$ & $\overset{\text{\tiny{DA}}}{\leftarrow}$ & $\overset{\text{\tiny{N}}}{\leftarrow}$ & $\overset{\text{\tiny{N}}}{\leftarrow}$\\ \hline
        $E$ & $\overset{\text{\tiny{N}}}{\leftarrow}$ & $\overset{\text{\tiny{DA}}}{\leftarrow}$ & $\overset{\text{\tiny{DA}}}{\leftarrow}$ & $\overset{\text{\tiny{N}}}{\leftarrow}$ & $\overset{\text{\tiny{N}}}{\leftarrow}$\\ \hline
      \end{tabular}
    \end{minipage}
    \begin{minipage}[b]{0.3\textwidth}
      \centering
      \begin{tabular}{|c|c|c|c|c|c|} \hline
        $\leftarrow$ & $A$ & $B$ & $C$ & $D$ & $E$\\ \hline
        $A$ & $\nparallel$ & $\nparallel$ & $\nparallel$ & $\nparallel$ & $\nparallel$\\ \hline
        $B$ & $\nparallel$ & $\nparallel$ & $\nparallel$ & $\nparallel$ & $\nparallel$\\ \hline
        $C$ & $\nparallel$ & $\nparallel$ & $\nparallel$ & $\nparallel$ & $\nparallel$\\ \hline
        $D$ & $\nparallel$ & $\nparallel$ & $\nparallel$ & $\nparallel$ & $\nparallel$\\ \hline
        $E$ & $\nparallel$ & $\nparallel$ & $\nparallel$ & $\nparallel$ & $\nparallel$\\ \hline
      \end{tabular}
    \end{minipage}
  \end{subtable}

  \begin{subtable}{1\textwidth}
    \vspace{1em}
    \centering
    \caption{图\ref{fig:nfc_example_2}中模型的ExRORU矩阵}
    \label{tab:nfc_example_2}
    \begin{minipage}[b]{0.3\textwidth}
      \centering
      \begin{tabular}{|c|c|c|c|c|c|} \hline
        $\rightarrow$ & $A$ & $B$ & $C$ & $D$ & $E$\\ \hline
        $A$ & $\overset{\text{\tiny{N}}}{\rightarrow}$ & $\overset{\text{\tiny{N}}}{\rightarrow}$ & $\overset{\text{\tiny{DA}}}{\rightarrow}$ & $\overset{\text{\tiny{DS}}}{\rightarrow}$ & $\overset{\text{\tiny{DS}}}{\rightarrow}$\\ \hline
        $B$ & $\overset{\text{\tiny{N}}}{\rightarrow}$ & $\overset{\text{\tiny{N}}}{\rightarrow}$ & $\overset{\text{\tiny{DA}}}{\rightarrow}$ & $\overset{\text{\tiny{DS}}}{\rightarrow}$ & $\overset{\text{\tiny{DS}}}{\rightarrow}$\\ \hline
        $C$ & $\overset{\text{\tiny{N}}}{\rightarrow}$ & $\overset{\text{\tiny{N}}}{\rightarrow}$ & $\overset{\text{\tiny{N}}}{\rightarrow}$ & $\overset{\text{\tiny{DS}}}{\rightarrow}$ & $\overset{\text{\tiny{DS}}}{\rightarrow}$\\ \hline
        $D$ & $\overset{\text{\tiny{N}}}{\rightarrow}$ & $\overset{\text{\tiny{N}}}{\rightarrow}$ & $\overset{\text{\tiny{N}}}{\rightarrow}$ & $\overset{\text{\tiny{N}}}{\rightarrow}$ & $\overset{\text{\tiny{N}}}{\rightarrow}$\\ \hline
        $E$ & $\overset{\text{\tiny{N}}}{\rightarrow}$ & $\overset{\text{\tiny{N}}}{\rightarrow}$ & $\overset{\text{\tiny{N}}}{\rightarrow}$ & $\overset{\text{\tiny{N}}}{\rightarrow}$ & $\overset{\text{\tiny{N}}}{\rightarrow}$\\ \hline
      \end{tabular}
    \end{minipage}
    \begin{minipage}[b]{0.3\textwidth}
      \centering
      \begin{tabular}{|c|c|c|c|c|c|} \hline
        $\leftarrow$ & $A$ & $B$ & $C$ & $D$ & $E$\\ \hline
        $A$ & $\overset{\text{\tiny{N}}}{\leftarrow}$ & $\overset{\text{\tiny{N}}}{\leftarrow}$ & $\overset{\text{\tiny{N}}}{\leftarrow}$ & $\overset{\text{\tiny{N}}}{\leftarrow}$ & $\overset{\text{\tiny{N}}}{\leftarrow}$\\ \hline
        $B$ & $\overset{\text{\tiny{N}}}{\leftarrow}$ & $\overset{\text{\tiny{N}}}{\leftarrow}$ & $\overset{\text{\tiny{N}}}{\leftarrow}$ & $\overset{\text{\tiny{N}}}{\leftarrow}$ & $\overset{\text{\tiny{N}}}{\leftarrow}$\\ \hline
        $C$ & $\overset{\text{\tiny{DS}}}{\leftarrow}$ & $\overset{\text{\tiny{DS}}}{\leftarrow}$ & $\overset{\text{\tiny{N}}}{\leftarrow}$ & $\overset{\text{\tiny{N}}}{\leftarrow}$ & $\overset{\text{\tiny{N}}}{\leftarrow}$\\ \hline
        $D$ & $\overset{\text{\tiny{DS}}}{\leftarrow}$ & $\overset{\text{\tiny{DS}}}{\leftarrow}$ & $\overset{\text{\tiny{DA}}}{\leftarrow}$ & $\overset{\text{\tiny{N}}}{\leftarrow}$ & $\overset{\text{\tiny{N}}}{\leftarrow}$\\ \hline
        $E$ & $\overset{\text{\tiny{DS}}}{\leftarrow}$ & $\overset{\text{\tiny{DS}}}{\leftarrow}$ & $\overset{\text{\tiny{DA}}}{\leftarrow}$ & $\overset{\text{\tiny{N}}}{\leftarrow}$ & $\overset{\text{\tiny{N}}}{\leftarrow}$\\ \hline
      \end{tabular}
    \end{minipage}
    \begin{minipage}[b]{0.3\textwidth}
      \centering
      \begin{tabular}{|c|c|c|c|c|c|} \hline
        $\leftarrow$ & $A$ & $B$ & $C$ & $D$ & $E$\\ \hline
        $A$ & $\nparallel$ & $\nparallel$ & $\nparallel$ & $\nparallel$ & $\nparallel$\\ \hline
        $B$ & $\nparallel$ & $\nparallel$ & $\nparallel$ & $\nparallel$ & $\nparallel$\\ \hline
        $C$ & $\nparallel$ & $\nparallel$ & $\nparallel$ & $\nparallel$ & $\nparallel$\\ \hline
        $D$ & $\nparallel$ & $\nparallel$ & $\nparallel$ & $\nparallel$ & $\nparallel$\\ \hline
        $E$ & $\nparallel$ & $\nparallel$ & $\nparallel$ & $\nparallel$ & $\nparallel$\\ \hline
      \end{tabular}
    \end{minipage}
  \end{subtable}
\end{table}

\section{本章小结}