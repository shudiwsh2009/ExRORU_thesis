\thusetup{
  %******************************
  % 注意:
  %   1. 配置里面不要出现空行
  %   2. 不需要的配置信息可以删除
  %******************************
  %
  %=====
  % 秘级
  %=====
  secretlevel={秘密},
  secretyear={10},
  %
  %=========
  % 中文信息
  %=========
  ctitle={基于精炼有序关系的流程模型行为语义刻画方法},
  cdegree={工程硕士专业},
  cdepartment={软件学院},
  cmajor={软件工程},
  cauthor={汪抒浩},
  csupervisor={王建民教授},
  % cassosupervisor={陈文光教授}, % 副指导老师
  % ccosupervisor={某某某教授}, % 联合指导老师
  % 日期自动使用当前时间,若需指定按如下方式修改:
  % cdate={超新星纪元},
  %
  % 博士后专有部分
  cfirstdiscipline={计算机科学与技术},
  cseconddiscipline={系统结构},
  postdoctordate={2009年7月——2011年7月},
  id={编号}, % 可以留空: id={},
  udc={UDC}, % 可以留空
  catalognumber={分类号}, % 可以留空
  %
  %=========
  % 英文信息
  %=========
  etitle={ExRORU: a New Approach to Characterize the Behavioral Semantics of Process Models},
  % 这块比较复杂,需要分情况讨论:
  % 1. 学术型硕士
  %    edegree:必须为Master of Arts或Master of Science(注意大小写)
  %             “哲学、文学、历史学、法学、教育学、艺术学门类,公共管理学科
  %              填写Master of Arts,其它填写Master of Science”
  %    emajor:“获得一级学科授权的学科填写一级学科名称,其它填写二级学科名称”
  % 2. 专业型硕士
  %    edegree:“填写专业学位英文名称全称”
  %    emajor:“工程硕士填写工程领域,其它专业学位不填写此项”
  % 3. 学术型博士
  %    edegree:Doctor of Philosophy(注意大小写)
  %    emajor:“获得一级学科授权的学科填写一级学科名称,其它填写二级学科名称”
  % 4. 专业型博士
  %    edegree:“填写专业学位英文名称全称”
  %    emajor:不填写此项
  edegree={Master of Engineering},
  emajor={Software Engineering},
  eauthor={Wang Shuhao},
  esupervisor={Professor Wang Jianmin},
  % eassosupervisor={Chen Wenguang},
  % 日期自动生成,若需指定按如下方式修改:
  % edate={December, 2005}
  %
  % 关键词用“英文逗号”分割
  ckeywords={流程模型, 不确定性精炼有序关系, 完全前缀展开, 行为语义},
  ekeywords={Process Model, Refined Ordering Relations with Uncertainty, Complete Prefix Unfolding, Behavioral Semantics}
}

% 定义中英文摘要和关键字
\begin{cabstract}
  TODO:这里是中文摘要
\end{cabstract}

% 如果习惯关键字跟在摘要文字后面,可以用直接命令来设置,如下:
% \ckeywords{\TeX, \LaTeX, CJK, 模板, 论文}

\begin{eabstract}
A recent paper has proposed new ordering relations with uncertainty between the executions of tasks in acyclic process models. This work also gave an algorithm to compute the relations between pairs of tasks based on complete prefix unfolding. However, this algorithm has serious limitations. It cannot work for cyclic process models and those with silent transitions and non-free-choice constructs. In practice most non-trivial process models contain cycles and about 10\% to 20\% have also non-free choice constructs. In this paper, we show how to overcome these problems by a refinement of the relations (i.e., extended refined ordering relations with uncertainty, ExRORU for short). All these relations can uniquely detect the behavioral difference between any pair of process models and can also be computed efficiently based on the complete prefix unfolding of a process. Experiments on real-life and synthesized models show that ExRORU is both effective and scalable.
\end{eabstract}

% \ekeywords{\TeX, \LaTeX, CJK, template, thesis}
