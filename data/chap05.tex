% !TEX root = ../main.tex
% !TEX program = xelatex

\chapter{总结与展望}\label{cha:conclusion}

\section{本文的工作总结}\label{sec:conclusion}
企业中海量过程模型的出现对于业务过程管理提出了新的挑战,过程模型仓库中的检索成为了一种关键技术。过程模型行为语义的刻画作为过程检索的基础,成为了不可或缺的研究方向。已有的过程模型行为语义刻画算法都存在一定的局限和不足,因此本文基于已有的RORU算法工作,提出了基于任务间扩展不确定性精炼有序关系的过程模型行为语义刻画方法,即ExRORU算法。本文使用完全前缀展开,即CPU技术高效计算任务间的ExRORU关系并以矩阵形式展现。ExRORU算法主要包含以下工作:
\begin{enumerate}[1.]
  \item 本文提出了过程流的概念,并基于过程流定义了三类事件间ExRORU关系,即因果关系、逆因果关系和并行关系,并通过引入直接和间接因果关系及不确定性将其进一步精炼成13种关系。随后本文给出了三类关系的图形化抽象形式,用“Skip”和“Loop”结构形象展示ExRORU关系中的不确定性。在这些抽象形式的启发下,本文分别给出了基于CPU高效计算事件间因果关系和并行关系的算法。事实证明,相比于基于模型完整轨迹集合或完整过程流集合,使用CPU计算事件间ExRORU关系更为实际且高效。
  \item 为了解决一个WF-net中的变迁可能对应CPU中多个事件的问题,本文定义了三类变迁间ExRORU关系。在获取WF-net中变迁间ExRORU关系时,需要逐一检查变迁的所有对应事件与其他变迁的对应事件间的关系,并将不同的事件间ExRORU关系折叠得到变迁间ExRORU关系。与事件间ExRORU关系类似,本文将变迁间ExRORU关系也进一步精炼为13种关系。有了这13种关系的帮助,ExRORU算法可以精确描述一个过程模型中任务执行间的关系。
  \item 本文形式化证明了ExRORU算法在描述过程模型行为语义时的唯一性,同时ExRORU算法可以检测一对过程模型行为语义间的细微差别。本文也证明了ExRORU算法中13种精炼关系的充分性和必要性。在与本文基础RORU算法的对比过程中,通过关系对比和实例验证说明了ExRORU算法改进了RORU算法的诸项缺点。
\end{enumerate}

本文在实验部分从有效性、性能和扩展性上对ExRORU算法进行了全面评估。实验结果表明,本文算法可以处理含有多种复杂结构的模型且有着可接受的性能表现。值得一提的是,ExRORU算法可以正确处理含有不可见变迁、非自由选择结构和循环结构的过程模型,且可以区别任意含有不同行为语义的过程模型对。

\section{未来工作展望}\label{sec:future_work}
通过本文第\ref{cha:experiment}章的实验以及分析可知,本文提出的ExRORU算法的计算效率相比于其他主流算法还不够优秀。同时,ExRORU作为一种可以唯一刻画过程模型行为语义的算法,可以被应用到过程模型检索、相似性度量等领域。未来ExRORU的工作可以从以下方面展开:
\begin{enumerate}[1.]
  \item 基于CPU计算ExRORU关系极大提高了计算效率,但是对于复杂模型来说在CPU上的计算仍然消耗大量时间。今后可以对于现有ExRORU计算方法做进一步改进,降低时间复杂度的同时给出形式化证明。
  \item ExRORU算法可以精确刻画过程模型的行为语义,可以被用于检索和相似性度量。因此,在ExRORU未来工作中,可以考虑基于ExRORU关系建议过程模型的索引从而用于检索过程,并基于现有的ExRORU矩阵量化过程模型之间的相似性。
\end{enumerate}