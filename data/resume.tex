% !TEX root = ../main.tex
% !TEX program = xelatex

\begin{resume}

  \resumeitem{个人简历}

  1992年7月28日出生于江西省贵溪市。

  2009年9月考入清华大学软件学院计算机软件专业,2013年7月本科毕业并获得工学学士学位。

  2013年9月免试进入清华大学软件学院攻读软件工程硕士学位至今。

  \researchitem{发表的学术论文} % 发表的和录用的合在一起

  \begin{publications}
    \item 汪抒浩, 闻立杰, 魏代森, 等. 基于任务最短跟随距离矩阵的流程模型行为相似性算法[J]. 计算机集成制造系统, 2013, 19(08): 1822-1831.(EI收录,检索号:20133916789912)
    \item Wang S, Yin M, Wang Z, et al. TAR++: A New Process Model Similarity Algorithm Based on the Importance of TARs[M]//Asia Pacific Business Process Management. Springer International Publishing, 2015: 98-112.(EI收录,检索号:20154601556389)
    \item Wang S, Lv C, Wen L, et al. Managing Massive Business Process Models and Instances with Process Space[C]//BPM (Demos). 2014: 91.
    \publicationskip
    \item Wang Z, Wen L, Wang J, et al. TAGER: transition-labeled graph edit distance similarity measure on process models[C]//On the Move to Meaningful Internet Systems: OTM 2014 Conferences. Springer Berlin Heidelberg, 2014: 184-201.(EI收录,检索号:20144700214651)
    \item 王子璇, 闻立杰, 汪抒浩, 等. 基于变迁标签图编辑距离的过程模型相似性度量[J]. 计算机集成制造系统, 2016, 22(2): 343-352.(EI收录,检索号:20161402175776)
  \end{publications}

  % \researchitem{研究成果} % 有就写,没有就删除
  % \begin{achievements}
  %   \item 任天令, 杨轶, 朱一平, 等. 硅基铁电微声学传感器畴极化区域控制和电极连接的
  %     方法: 中国, CN1602118A. (中国专利公开号)
  %   \item Ren T L, Yang Y, Zhu Y P, et al. Piezoelectric micro acoustic sensor
  %     based on ferroelectric materials: USA, No.11/215, 102. (美国发明专利申请号)
  % \end{achievements}

\end{resume}
