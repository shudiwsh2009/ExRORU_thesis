\begin{resume}

  \resumeitem{个人简历}

  1992年7月28日出生于江西省贵溪市。

  2009年9月考入清华大学软件学院计算机软件专业,2013年7月本科毕业并获得工学学士学位。

  2013年9月免试进入清华大学软件学院攻读软件工程硕士学位至今。

  \researchitem{发表的学术论文} % 发表的和录用的合在一起

  % 1. 已经刊载的学术论文(本人是第一作者,或者导师为第一作者本人是第二作者)
  \begin{publications}
    \item 汪抒浩, 闻立杰, 魏代森, 等. 基于任务最短跟随距离矩阵的流程模型行为相似性算法[J]. 计算机集成制造系统, 2013, 19(08): 1822-1831.(EI收录,检索号:20133916789912)
    \item Wang S, Yin M, Wang Z, et al. TAR++: A New Process Model Similarity Algorithm Based on the Importance of TARs[M]//Asia Pacific Business Process Management. Springer International Publishing, 2015: 98-112.(EI收录,检索号:20154601556389)
    \item Wang S, Lv C, Wen L, et al. Managing Massive Business Process Models and Instances with Process Space[C]//BPM (Demos). 2014: 91.
  \end{publications}

  % % 2. 尚未刊载,但已经接到正式录用函的学术论文(本人为第一作者,或者
  % %    导师为第一作者本人是第二作者)。
  % \begin{publications}[before=\publicationskip,after=\publicationskip]
  %   \item Yang Y, Ren T L, Zhu Y P, et al. PMUTs for handwriting recognition. In
  %     press. (已被 Integrated Ferroelectrics 录用. SCI 源刊.)
  % \end{publications}

  % % 3. 其他学术论文。可列出除上述两种情况以外的其他学术论文,但必须是
  % %    已经刊载或者收到正式录用函的论文。
  % \begin{publications}
  %   \item Wu X M, Yang Y, Cai J, et al. Measurements of ferroelectric MEMS
  %     microphones. Integrated Ferroelectrics, 2005, 69:417-429. (SCI 收录, 检索号
  %     :896KM)
  %   \item 贾泽, 杨轶, 陈兢, 等. 用于压电和电容微麦克风的体硅腐蚀相关研究. 压电与声
  %     光, 2006, 28(1):117-119. (EI 收录, 检索号:06129773469)
  %   \item 伍晓明, 杨轶, 张宁欣, 等. 基于MEMS技术的集成铁电硅微麦克风. 中国集成电路,
  %     2003, 53:59-61.
  % \end{publications}

  % \researchitem{研究成果} % 有就写,没有就删除
  % \begin{achievements}
  %   \item 任天令, 杨轶, 朱一平, 等. 硅基铁电微声学传感器畴极化区域控制和电极连接的
  %     方法: 中国, CN1602118A. (中国专利公开号)
  %   \item Ren T L, Yang Y, Zhu Y P, et al. Piezoelectric micro acoustic sensor
  %     based on ferroelectric materials: USA, No.11/215, 102. (美国发明专利申请号)
  % \end{achievements}

\end{resume}
